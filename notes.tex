\section{Week 1}
\begin{itemize}
\item The memory is divided in bytes and 1 byte is equal to 8 bits. To store integer we use a unit of 4 bytes if processor is 32bits or 8 bytes if it is 64bits. 
\item When python shows the letter L in the end of a integer, it is indicating that this number needs more than one memory's unit (4 or 8 bytes) to be stored and you have to lead with this.
\item float number,$0.23 * 10^3$, 0.23 is the mantissa and 3 is the expoent. 
\end{itemize}

\subsection{Newton's Method}
To find a solution for $\sqrt{A}$, we look for a function that has $\sqrt{A}$ as root, for example, $f(x)=x^2-A$. The roots are: $-\sqrt{A}$ and $+\sqrt{A}$. Using the Newton's Method:

\begin{equation}
  x_{n+1} = x_n + \frac{f(x_n)}{f\prime(x_n)} 
\end{equation}

Applying in our function: 
\begin{equation}
  x_{n+1} = \frac{x_n}{2} + \frac{A}{2x_n}
\end{equation}

\subsection{python}
\begin{itemize}
  \item apt-get install ipython-notebook python-scipy python-numpy python-matplotlib
  \item ipython notebook --pylab inline
\end{itemize}

In python we can use lists and tuples. lists are $[1,'abc',3.4,[4,5]]$ and tuple $(1,'abc',3.4)$. 
Both accept differents kinds of objects, but list is mutable and tuple is immutable. Mutable objects point to the same location at memory. We can not assign a new value to a position of tuple, like $mytuple[2] = 'new'$
We can lead list as mutable objects, to assign their values: 
new-list = list(myoldlist). To verify position on memory on python: id(mylist).
\subsection{Peformance}
\begin{itemize}
  \item Latency: amount of time to complete a given work
  \item Throughput: amount of work to be completed per unit of time
\end{itemize}

The idea is hide latency and improve throughput

\subsection{Parallel Computing}
OpemMP: shared memory \\
MPI: distributed memory 

Amdahl\' Law: only part of computation can be parallelized.

Example: Suppose 10\% of the computation is inherently sequential and the 90\% is parallelized, how much faster could the computation run on many processors? at a factor of 10.

Law: $1/S$ of the computation is inherently sequential and the $(1-1/S)$ is parallelized, how much faster could the computation run on many processors? the can gain at most a factor of S, no matter how many processors.

\begin{equation}
T_p = (1/S)T_S + (1-1/S)T_S/P 
\end{equation}
P: number of processors. Tp: time to run the program. Ts: time required in a sequential machine.

speedup: $T_S/T_P$, normally less than P.

omp critical: section is used to denote a section of parallel code that should only be executed by one thread at a time. 

A "private clause" on an OpenMP directive is used to indicate variables that must be created for each thread as distinct memory locations.

imagemagick provides the command display to see images on remote server. 

In ssh connections use -X parameter.

