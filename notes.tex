\section{Week 1}
\begin{itemize}
\item The memory is divided in bytes and 1 byte is equal to 8 bits. To store integer we use a unit of 4 bytes if processor is 32bits or 8 bytes if it is 64bits. 
\item When python shows the letter L in the end of a integer, it is indicating that this number needs more than one memory's unit (4 or 8 bytes) to be stored and you have to lead with this.
\item float number,$0.23 * 10^3$, 0.23 is the mantissa and 3 is the expoent. 
\end{itemize}

\subsection{Newton's Method}
To find a solution for $\sqrt{A}$, we look for a function that has $\sqrt{A}$ as root, for example, $f(x)=x^2-A$. The roots are: $-\sqrt{A}$ and $+\sqrt{A}$. Using the Newton's Method:

\begin{equation}
  x_{n+1} = x_n + \frac{f(x_n)}{f\prime(x_n)} 
\end{equation}

Applying in our function: 
\begin{equation}
  x_{n+1} = \frac{x_n}{2} + \frac{A}{2x_n}
\end{equation}

\subsection{python}
\begin{itemize}
  \item apt-get install ipython-notebook python-scipy python-numpy python-matplotlib
  \item ipython notebook --pylab inline
\end{itemize}

In python we can use lists and tuples. lists are $[1,'abc',3.4,[4,5]]$ and tuple $(1,'abc',3.4)$. 
Both accept differents kinds of objects, but list is mutable and tuple is immutable. Mutable objects point to the same location at memory. We can not assign a new value to a position of tuple, like $mytuple[2] = 'new'$
We can lead list as mutable objects, to assign their values: 
new-list = list(myoldlist). To verify position on memory on python: id(mylist).

